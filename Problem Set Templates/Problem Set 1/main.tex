\documentclass[10pt]{article}

%Now, we can import whatever packages we may need
%These are the ones I commonly utilize.
\usepackage[english]{babel}
\usepackage{amsmath} %Contains useful commands like \text and equation environment
\usepackage{amsthm} %Helps to define theorem structures, includes \newtheorem and proof environment
\usepackage{amssymb} %Provides an extended symbol collection
\usepackage{graphicx} %Allows us to display figures
\usepackage{fancyvrb} %Contains the Verbatim environment, allowing us to type code blocks
\usepackage{fancyhdr} %Allows us to customize headers and footers
\usepackage{lastpage} %Gives us the last page as a command \pageref{LastPage}
\usepackage{hyperref} %Allows us to have hyper links in our text, \href
\usepackage{cleveref} %Fancier version of reference by using \cref
\usepackage[labelformat=empty]{caption}
\usepackage{algpseudocode}
\usepackage{algorithm}
\usepackage{tikz}
\usepackage{witharrows}

%Here we create some useful shortcuts for fancy characters and spacing
\newcommand{\bb}[1]{\mathbb{#1}} %Shortcut for bold faced characters
\newcommand{\Z}{\bb{Z}} %Returns the set of Integers symbol
\newcommand{\Q}{\bb{Q}} %Returns set of rational numbers symbol
\newcommand{\R}{\bb{R}} %Real numbers symbol
\newcommand{\C}{\bb{C}} %Complex numbers symbol
\newcommand{\N}{\bb{N}} %Natural numbers symbol
\newcommand{\obar}[1]{\overline{#1}} %Shortcut for putting an overline over a symbol for equivalence classes
\newcommand{\dent}{\hspace{\parindent}} %Forces an indent in the current line

%Adjust our page setup
\setlength{\topmargin}{-0.3in} %Removes some top margin so our header and footer begins higher up
\setlength{\oddsidemargin}{0in} %Custom margin for even and odd pages
\setlength{\evensidemargin}{0in} %Used when printing double-sided papers 
\setlength{\textheight}{9in} %Leaves 1 inch at top and 1 inch at bottom
\setlength{\textwidth}{6.5in} %Leaves 3/4 inch on each side
\setlength{\parindent}{20pt} %Here you can customize a paragraph indentation

%Set up our headers and footers
\pagestyle{fancy} %Sets our page style using fancyhdr
\lhead{Math 735} %Left Header
\chead{Homework \#1}
\rhead{\today\\Name Here} %Right Header
\cfoot{\thepage \ of \pageref{LastPage}} %Center Footer
\renewcommand{\footrulewidth}{0.4pt} %Enables horizontal line above the footer

%Commands for theorem environments
\newtheorem{theorem}{Theorem}

%Custom Theorem Numbering
\newenvironment{customthm}[1]
  {\renewcommand\thetheorem{#1}\theorem}
  {\endtheorem}

\setcounter{totalnumber}{4}

\setlength{\headheight}{22.54448pt}

%Set up our hyperlink settings
\hypersetup{colorlinks, citecolor=black, filecolor=black, linkcolor=black, urlcolor=blue}

\begin{document}

    \begin{center}
        \section*{0.2 Properties of the Integers}
    \end{center}

    \begin{itemize}
        \item[6.] Prove the Well Ordering Property of \(\Z\) by induction and prove the minimal element is unique.

        \begin{customthm}{}
            (Well Ordering of \(\Z\)) If \(A\) is any nonempty subset of \(\Z^+\), there is some element \(m\in A\) such that \(m\leq a\), for all \(a\in A\) (\(m\) is called the \emph{minimal} element of \(A\).
        \end{customthm}

        \begin{proof}
            %%%%%%%%%%%%%%%%%%%%%%
            % TYPE UP PROOF HERE %
            %%%%%%%%%%%%%%%%%%%%%%
        \end{proof}
        
    \end{itemize}

    \begin{itemize}
        \item[11.] Prove that if \(d\) divides \(n\) then \(\varphi(d)\) divides \(\varphi(n)\) where \(\varphi\) denotes Euler's \(\varphi\)-function.

        \begin{proof}
            %%%%%%%%%%%%%%%%%%%%%%
            % TYPE UP PROOF HERE %
            %%%%%%%%%%%%%%%%%%%%%%
        \end{proof}
        
    \end{itemize}

    \begin{center}
        \section*{0.3 \(\Z/n\Z\,:\,\)The Integers Modulo \(n\)}
    \end{center}

    \begin{itemize}
        \item[10.] Prove that the number of elements of \((\Z/n\Z)^\times\) is \(\varphi(n)\) where \(\varpi\) denotes the Euler \(\varphi\)-function.

        \begin{proof}
            %%%%%%%%%%%%%%%%%%%%%%
            % TYPE UP PROOF HERE %
            %%%%%%%%%%%%%%%%%%%%%%
        \end{proof}
        
    \end{itemize}

    \begin{itemize}
        \item[11.]Prove that if \(\obar{a},\obar{b}\in(\Z/n\Z)^\times\), then \(\obar{a}\cdot\obar{b}\in(\Z/n\Z)^\times\).

        \begin{proof}
            %%%%%%%%%%%%%%%%%%%%%%
            % TYPE UP PROOF HERE %
            %%%%%%%%%%%%%%%%%%%%%%
        \end{proof}
        
    \end{itemize}

    \noindent\textbf{How did you get interested in math?}

    Answer Here

    \vspace{5pt}

    \noindent\textbf{What are your goals for the course and for the semester?}

    Answer Here

    \vspace{5pt}

    \noindent\textbf{Tell me about someone you met in this course. (This will require you to meet someone - preferably new!- in the course outside of class to strike up a conversation.)}

    Answer Here

\end{document}