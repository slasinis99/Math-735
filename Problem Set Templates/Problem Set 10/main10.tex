\documentclass[10pt]{article}

% Standard LaTeX Packages
\usepackage[english]{babel}
\usepackage{graphicx}

% Normal Math Packages
\usepackage{amsmath}
\usepackage{amssymb}
\usepackage{amsthm}

% Packages that allow me to write algorithms should the need arise.
\usepackage{algpseudocode}
\usepackage{algorithm2e}

% Code blocks, anyone?
\usepackage{fancyvrb}

% Slightly degenerate normal use packages.
\usepackage{fancyhdr}
\usepackage{hyperref}
\usepackage{cleveref}
\usepackage{setspace}
\usepackage{lastpage}

% Slightly degenerate math packages.
\usepackage{tikz}
\usepackage{witharrows}

% Very niche, degenerate packages.
\usepackage{xparse}
\usepackage{tcolorbox}

% Paging Stuff
\setlength{\textwidth}{6.5in}
\setlength{\textheight}{9in}
\setlength{\oddsidemargin}{0in}
\setlength{\evensidemargin}{0in}
\setlength{\topmargin}{-0.25in}

% Headers and Footers
\pagestyle{fancy}
\fancyhead{}
\fancyhead[L]{Mathematics 735}
\fancyhead[C]{Problem Set 10}
\fancyhead[R]{20 November, 2024\\Tyroil Smoochie-Wallace}
\fancyfoot{}
\fancyfoot[C]{\thepage \ of \pageref{LastPage}}
\renewcommand{\footrulewidth}{0.4pt}

% Double spacing, should your professor request it.
\doublespacing

% Set up how links appear.
\hypersetup{
    colorlinks,
    citecolor=black,
    filecolor=black,
    linkcolor=black,
    urlcolor=blue
}

% Gets rid of annoying yellow warning for \headheight
\setlength{\headheight}{22.54448pt}

% Math Blackboard Bold Letters (i.e., the real numbers)
\newcommand{\bb}[1]{\mathbb{#1}}
\newcommand{\A}{\mathbb{A}}
\newcommand{\B}{\mathbb{B}}
\newcommand{\C}{\mathbb{C}}
\newcommand{\D}{\mathbb{D}}
\newcommand{\E}{\mathbb{E}}
\newcommand{\F}{\mathbb{F}}
\newcommand{\G}{\mathbb{G}}
\newcommand{\I}{\mathbb{I}}
\newcommand{\J}{\mathbb{J}}
\newcommand{\K}{\mathbb{K}}
\newcommand{\M}{\mathbb{M}}
\newcommand{\N}{\mathbb{N}}
\newcommand{\Q}{\mathbb{Q}}
\newcommand{\R}{\mathbb{R}}
\newcommand{\T}{\mathbb{T}}
\newcommand{\U}{\mathbb{U}}
\newcommand{\V}{\mathbb{V}}
\newcommand{\W}{\mathbb{W}}
\newcommand{\X}{\mathbb{X}}
\newcommand{\Y}{\mathbb{Y}}
\newcommand{\Z}{\mathbb{Z}}

% Group Action Symbol
\newcommand{\acts}{\reflectbox{\righttoleftarrow}}

% Math Operators
\DeclareMathOperator{\im}{Im}
\DeclareMathOperator{\nsg}{\trianglelefteq}

% Extremely degenerate command for cycle notation.
\ExplSyntaxOn
\NewDocumentCommand{\cycle}{ O{\;} m }
 {
  (
  \alec_cycle:nn { #1 } { #2 }
  )
 }

\seq_new:N \l_alec_cycle_seq
\cs_new_protected:Npn \alec_cycle:nn #1 #2
 {
  \seq_set_split:Nnn \l_alec_cycle_seq { , } { #2 }
  \seq_use:Nn \l_alec_cycle_seq { #1 }
 }
\ExplSyntaxOff

% Custom Problem Environment
\newtcolorbox[auto counter, number within=section]{problem}[2][]{colback=gray!5!white, colframe=gray!75!black, fonttitle=\bfseries, title=Problem~\thetcbcounter, #1}

\begin{document}

\setcounter{section}{6}
\section{Introduction to Rings}

\setcounter{subsection}{2}
\subsection{Ring Homomorphisms and Quotient Rings}

\begin{problem}[title=Problem 26]
    \par The \emph{characteristic} of a ring \(R\) is the smallest positive integer \(n\) such that \(1+1+\cdots+1=0\) (\(n\) times) in \(R\); if no such integer exists the characteristic of \(R\) is said to be \(0\). For example, \(\Z/n\Z\) is a ring of characteristic \(n\) for each positive integer \(n\) and \(\Z\) is a ring of characteristic \(0\).
    \begin{itemize}
        \item[(a)] Prove that the map \(\Z\to R\) defined by
        \[k\mapsto\begin{cases}1+1+\cdots+1\,(k \text{ times})&\text{if }k>0\\0&\text{if }k=0\\-1-1-\cdots-1\,(k\text{ times})&\text{if }k<0\end{cases}\]
        is a ring homomorphism whose kernel is \(n\Z\), where \(n\) is the characteristic of \(R\) (this explains the use of the terminology ``characteristic 0'' instead of the archaic phrase ``characteristic \(\infty\)'' for rings in which no sum of 1's is zero).
        \item[(b)] Determine the characteristics of the rings \(\Q,\Z[x],\Z/n\Z[x]\).
        \item[c)] Prove that if \(p\) is a prime and if \(R\) is a commutative ring of characteristic \(p\), then \((a+b)^p=a^p+b^p\) for all \(a,b\in R\).
    \end{itemize}
\end{problem}

\begin{problem}[title=Problem 29]
    \par Let \(R\) be a commutative ring. Recall (cf. Exercise 13, Section 1) than an element \(x\in R\) is nilpotent if \(x^n=0\) for some \(n\in\Z^+\). Prove that the set of nilpotent elements form an ideal---called the \emph{nilradical} of \(R\) and denoted \(\mathfrak{R}(R)\). [Use the Binomial Theorem to show \(\mathfrak{R}(R)\) is closed under addition.]
\end{problem}

\subsection{Properties of Ideals}

\begin{problem}[title=Problem 7]
    \par Let \(R\) be a commutative ring with 1. Prove that the principal ideal generated by \(x\) in the polynomial ring \(R[x]\) is a prime ideal if and only if \(R\) is an integral domain. Prove that \((x)\) is a maximal ideal if and only if \(R\) is a field.
\end{problem}

\begin{problem}[title=Problem 14]
    \par Assume \(R\) is commutative. Let \(x\) be an indeterminate, let \(f(x)\) be a monic polynomial in \(R[x]\) of degree \(n\geq 1\) and use the bar notation to denote passage to quotient ring \(R[x]/(f(x))\).
    \begin{itemize}
        \item[(a)] Show that every element of \(R[x]/(f(x))\) is of the form \(\overline{p(x)}\) for some polynomial \(p(x)\in R[x]\) f degree less than \(n\), i.e.,
        \[R[x]/(f(x))=\left\{\overline{a_0}+\overline{a_1x}+\cdots+\overline{a_{n-1}x^{n-1}}\mid a_0,a_1,\dots,a_{n-1}\in R\right\}.\]
        [If \(f(x)=x^n+b_{n-1}x^{n-1}+\cdots+b_0\) then \(\overline{x^n}=\overline{-(b_{n-1}x^{n-1}+\cdots+b_0}\). Use this to reduce powers of \(\overline{x}\) in the quotient ring.]
        \item[(b)] Prove that if \(p(x)\) and \(q(x)\) are distinct polynomials in \(R[x]\) which are both of degree less than \(n\), then \(\overline{p(x)}\neq\overline{q(x)}\). [Otherwise \(p(x)-q(x)\) is an \(R[x]\)-multiple of the monic polynomial \(f(x)\).]
        \item[(c)] If \(f(x)=a(x)b(x)\) where both \(a(x)\) and \(b(x)\) have degree less than \(n\), prove that \(\overline{a(x)}\) is a zero divisor in \(R[x]/(f(x))\).
        \item[(d)] If \(f(x)=x^n-a\) for some nilpotent element \(a\in R\), prove that \(\overline{x}\) is nilpotent in \(R[x]/(f(x))\).
        \item[(e)] Let \(p\) be a prime, assume \(R=\F_p\) and \(f(x)=x^p-a\) for some \(a\in\F_p\). Prove that \(\overline{x-a}\) is nilpotent in \(R[x]/(f(x))\). [Use Exercise 26(c) of Section 3.]
    \end{itemize}
\end{problem}

\section*{Compulsory Supplemental Exercise(s)}

\begin{problem}[title=Problem 3]
    \par Let \(\phi\,:\,R\to S\) be a ring homomorphism. Let \(P\) be a prime ideal of \(R\) and \(Q\) be a prime ideal of \(S\).
    \begin{itemize}
        \item[(a)] Is \(\phi(P)\) and ideal of \(S\)?
        \item[(b)] Is \(\phi(P)\) a prime ideal of \(S\)?
        \item[(c)] Is \(\phi^{-1}(Q)\) an ideal of \(R\)?
        \item[(d)] Is \(\phi^{-1}(Q)\) a prime ideal of \(R\)?
        \item[(e)] Think about the questions from (a)-(d). What happens in \(\phi\) is surjective?
    \end{itemize}
\end{problem}

\appendix

\section{Collaborators}

\begin{itemize}
    \item 
\end{itemize}
    
\end{document}